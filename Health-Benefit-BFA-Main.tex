% Options for packages loaded elsewhere
\PassOptionsToPackage{unicode}{hyperref}
\PassOptionsToPackage{hyphens}{url}
%
\documentclass[
]{article}
\usepackage{lmodern}
\usepackage{amssymb,amsmath}
\usepackage{ifxetex,ifluatex}
\ifnum 0\ifxetex 1\fi\ifluatex 1\fi=0 % if pdftex
  \usepackage[T1]{fontenc}
  \usepackage[utf8]{inputenc}
  \usepackage{textcomp} % provide euro and other symbols
\else % if luatex or xetex
  \usepackage{unicode-math}
  \defaultfontfeatures{Scale=MatchLowercase}
  \defaultfontfeatures[\rmfamily]{Ligatures=TeX,Scale=1}
\fi
% Use upquote if available, for straight quotes in verbatim environments
\IfFileExists{upquote.sty}{\usepackage{upquote}}{}
\IfFileExists{microtype.sty}{% use microtype if available
  \usepackage[]{microtype}
  \UseMicrotypeSet[protrusion]{basicmath} % disable protrusion for tt fonts
}{}
\makeatletter
\@ifundefined{KOMAClassName}{% if non-KOMA class
  \IfFileExists{parskip.sty}{%
    \usepackage{parskip}
  }{% else
    \setlength{\parindent}{0pt}
    \setlength{\parskip}{6pt plus 2pt minus 1pt}}
}{% if KOMA class
  \KOMAoptions{parskip=half}}
\makeatother
\usepackage{xcolor}
\IfFileExists{xurl.sty}{\usepackage{xurl}}{} % add URL line breaks if available
\IfFileExists{bookmark.sty}{\usepackage{bookmark}}{\usepackage{hyperref}}
\hypersetup{
  pdftitle={BFA Burden of Health Calculations},
  pdfauthor={Alon Shepon},
  hidelinks,
  pdfcreator={LaTeX via pandoc}}
\urlstyle{same} % disable monospaced font for URLs
\usepackage[margin=1in]{geometry}
\usepackage{color}
\usepackage{fancyvrb}
\newcommand{\VerbBar}{|}
\newcommand{\VERB}{\Verb[commandchars=\\\{\}]}
\DefineVerbatimEnvironment{Highlighting}{Verbatim}{commandchars=\\\{\}}
% Add ',fontsize=\small' for more characters per line
\usepackage{framed}
\definecolor{shadecolor}{RGB}{248,248,248}
\newenvironment{Shaded}{\begin{snugshade}}{\end{snugshade}}
\newcommand{\AlertTok}[1]{\textcolor[rgb]{0.94,0.16,0.16}{#1}}
\newcommand{\AnnotationTok}[1]{\textcolor[rgb]{0.56,0.35,0.01}{\textbf{\textit{#1}}}}
\newcommand{\AttributeTok}[1]{\textcolor[rgb]{0.77,0.63,0.00}{#1}}
\newcommand{\BaseNTok}[1]{\textcolor[rgb]{0.00,0.00,0.81}{#1}}
\newcommand{\BuiltInTok}[1]{#1}
\newcommand{\CharTok}[1]{\textcolor[rgb]{0.31,0.60,0.02}{#1}}
\newcommand{\CommentTok}[1]{\textcolor[rgb]{0.56,0.35,0.01}{\textit{#1}}}
\newcommand{\CommentVarTok}[1]{\textcolor[rgb]{0.56,0.35,0.01}{\textbf{\textit{#1}}}}
\newcommand{\ConstantTok}[1]{\textcolor[rgb]{0.00,0.00,0.00}{#1}}
\newcommand{\ControlFlowTok}[1]{\textcolor[rgb]{0.13,0.29,0.53}{\textbf{#1}}}
\newcommand{\DataTypeTok}[1]{\textcolor[rgb]{0.13,0.29,0.53}{#1}}
\newcommand{\DecValTok}[1]{\textcolor[rgb]{0.00,0.00,0.81}{#1}}
\newcommand{\DocumentationTok}[1]{\textcolor[rgb]{0.56,0.35,0.01}{\textbf{\textit{#1}}}}
\newcommand{\ErrorTok}[1]{\textcolor[rgb]{0.64,0.00,0.00}{\textbf{#1}}}
\newcommand{\ExtensionTok}[1]{#1}
\newcommand{\FloatTok}[1]{\textcolor[rgb]{0.00,0.00,0.81}{#1}}
\newcommand{\FunctionTok}[1]{\textcolor[rgb]{0.00,0.00,0.00}{#1}}
\newcommand{\ImportTok}[1]{#1}
\newcommand{\InformationTok}[1]{\textcolor[rgb]{0.56,0.35,0.01}{\textbf{\textit{#1}}}}
\newcommand{\KeywordTok}[1]{\textcolor[rgb]{0.13,0.29,0.53}{\textbf{#1}}}
\newcommand{\NormalTok}[1]{#1}
\newcommand{\OperatorTok}[1]{\textcolor[rgb]{0.81,0.36,0.00}{\textbf{#1}}}
\newcommand{\OtherTok}[1]{\textcolor[rgb]{0.56,0.35,0.01}{#1}}
\newcommand{\PreprocessorTok}[1]{\textcolor[rgb]{0.56,0.35,0.01}{\textit{#1}}}
\newcommand{\RegionMarkerTok}[1]{#1}
\newcommand{\SpecialCharTok}[1]{\textcolor[rgb]{0.00,0.00,0.00}{#1}}
\newcommand{\SpecialStringTok}[1]{\textcolor[rgb]{0.31,0.60,0.02}{#1}}
\newcommand{\StringTok}[1]{\textcolor[rgb]{0.31,0.60,0.02}{#1}}
\newcommand{\VariableTok}[1]{\textcolor[rgb]{0.00,0.00,0.00}{#1}}
\newcommand{\VerbatimStringTok}[1]{\textcolor[rgb]{0.31,0.60,0.02}{#1}}
\newcommand{\WarningTok}[1]{\textcolor[rgb]{0.56,0.35,0.01}{\textbf{\textit{#1}}}}
\usepackage{graphicx,grffile}
\makeatletter
\def\maxwidth{\ifdim\Gin@nat@width>\linewidth\linewidth\else\Gin@nat@width\fi}
\def\maxheight{\ifdim\Gin@nat@height>\textheight\textheight\else\Gin@nat@height\fi}
\makeatother
% Scale images if necessary, so that they will not overflow the page
% margins by default, and it is still possible to overwrite the defaults
% using explicit options in \includegraphics[width, height, ...]{}
\setkeys{Gin}{width=\maxwidth,height=\maxheight,keepaspectratio}
% Set default figure placement to htbp
\makeatletter
\def\fps@figure{htbp}
\makeatother
\setlength{\emergencystretch}{3em} % prevent overfull lines
\providecommand{\tightlist}{%
  \setlength{\itemsep}{0pt}\setlength{\parskip}{0pt}}
\setcounter{secnumdepth}{-\maxdimen} % remove section numbering

\title{BFA Burden of Health Calculations}
\author{Alon Shepon}
\date{10/27/2020}

\begin{document}
\maketitle

\hypertarget{calculation-of-burden-of-disease-for-changes-in-fish-consumption}{%
\section{Calculation of Burden of disease for changes in fish
consumption}\label{calculation-of-burden-of-disease-for-changes-in-fish-consumption}}

The GBD project is the most comprehensive methodology to assess health
burden across countries, age and sex groups and various forms of disease
and risks. The GBD project has calculated the burden of malnutrition for
vitamin A, zinc, iron (and other nutrient deficiencies) and low seafood
consumption (PUFA-\textgreater heart disease) (Murray 2020, Afshin 2019
and James 2018) - four risk factors associated with fish. While not
comprehensive of all outcomes associated with consuming fish (in the
health literature) e.g.~omega n-3 impact on child development), it seems
to include both micronutrient contribution and low omega n-3 derived
from fish. Using the DALYs metrics, our results can be comapred to other
GBD global metrics results, enabling us to assess the magnitude of
health burdens associated with reduced or increased fish consumption in
future alternatives.

\hypertarget{calculating-fish-associated-burden-of-disease-in-2030}{%
\subsubsection{calculating fish-associated burden of disease in
2030}\label{calculating-fish-associated-burden-of-disease-in-2030}}

\begin{enumerate}
\def\labelenumi{\arabic{enumi}.}
\item
  Build GBD dataset. As a first step, we coalesced the GBD historical
  data for all countries per age-sex groups and for the specific risks
  and causes associated with fish consumption that exist in the GBD
  database, namely zinc, iron, vitamin A and low omega n-3 consumption.
\item
  Extract and upload population data from present till 2030 per age-sex
  country group.
\item
  Using that historical data we extrapolated the burden of disease in
  year 2030 (hereafter DALY2030) for each age-sex-location using a
  moving regression.
\end{enumerate}

\hypertarget{intake-distributions-i-per-age-sex-location-groups}{%
\subsubsection{intake distributions (I) per age-sex-location
groups}\label{intake-distributions-i-per-age-sex-location-groups}}

\begin{enumerate}
\def\labelenumi{\arabic{enumi}.}
\item
  Deriving national level distribution of micronutrients from SPADE:
  This stage will be done using the SPADE software
  (\url{https://www.rivm.nl/en/spade}) which analyzes existing HCES data
  and translates 24h recall into average distributions. Currently we
  will have estimates for a handful of countries, in all continents,
  from which we will infer on the distributions for all countries.
\item
  Deriving average micronutrient intakes from the Aglink Cosimo: This
  FAO model will simulate consumption of fish in future scenarios based
  on price elasticities, supply and demand curves and will output
  average nutrient intakes for reference and alternative futures per
  age-sex-country group.
\end{enumerate}

\hypertarget{deriving-age-sex-location-changes-in-dalys-of-micronutrient-deficiences-and-seafood-omega-n-3-burdens-due-to-perturbations-in-2030}{%
\subsubsection{Deriving age-sex-location changes in DALYs of
micronutrient deficiences and seafood omega n-3 burdens due to
perturbations in
2030}\label{deriving-age-sex-location-changes-in-dalys-of-micronutrient-deficiences-and-seafood-omega-n-3-burdens-due-to-perturbations-in-2030}}

We compute the changes in DALYs (changes in health burden) per
age-sex-country for each of the four risks as response to moving from
the reference scenario (ref) to the alternative scenario (alt) in year
2030 using the following equation: \[
                                    ∆DALY_{c,a,s,r}=DALY^{2030}_{c,a,s,r}*(SEV^{alt}_{c,a,s,r}/SEV^{ref}_{c,a,s,r}-1)
                                    \]

Where \(DALY_{2030,c,a,s,r}\) is the DALY per age (a), sex (s), country
(c) and risk (r) value for year 2030 derived above, and
\(SEV^{alt}_{c,a,s,r}\) and \(SEV^{ref}_{c,a,s,r}\) are the population
level average weighted exposure to risk for the alternative
(``perturbed'') and reference (``baseline'') scenarios, respectively.
SEVs are summary exposure values and they reflect excess average
weighted prevelance of exposure (or inadequancy in this case). This
value is equal to: \[
                         SEV^{ref/alt}_{c,a,s,r} = \int(I^{ref/alt}_{c,a,s,r}*RR_{c,a,s,r})\]

Usually SEV is derived by dividing the above term with the maximal risk
\(R_{max}\); however in our calculation the maximal risk is 1 (no
consumption of the micronutrient).As consumption increases, RR
effectively decreases. RR curves For omega n-3 can be derived from the
GBD source (either the GBD 2017 log-linear curve or the GBD 2020 spline)
or any other one we want to use (for example the Thomsen paper used the
Mozaffarian and Rimm 2006 curve and simplified it to a descending linear
curve, until a consumption level of 250 mg/cap/d EPA+DHA).

For the micronutrient calculation, I suggest two possible approaches. In
the first, we use GBD relative risk curves. A second approach can be
building the RR distributions by taking country level EAR values and
constructing a descending risk curve based on the CDF (cumulative
distribution function) of a normal distribution. IOM suggest using a CV
of 10\%-15\% when there is no sufficient data on the requirements.
Multiplying intake distribution with these RR curves and integrating
will result in the prevalence of inadequacy (probability method for
calculating deficiency), in effect a population-level risk (RR). As the
above reference indicates, as long as the requirement distribution is
symmetrical (not for iron) the results are insensitive to the shape and
SD of the requirement curve. As for intakes, these are derived by taking
the average nutrients/PUFA from the FAO model (for both scenarios) and
building a distributions around them based on the shape deduced from
SPADE; using Monte Carlo we can then assess how changes in the tail of
this distribution affect the overall calcs.

Overall health burden resulting from the perturbation in consumption
will be the sum of all burdens examined for each country per age-sex
groups: \[∑_{r}∆DALY_{a,s,c,r}.\]

This approach allows comparison across the different risks and assess
their relative contribution to the total health burden following a
shock. For example in developing countries one might expect that the
DALY values from micronutrient deficiencies will be sensitive to shocks
more than in developed countries and possibly on par with changes in
EPA+DHA contribution to health.

\begin{enumerate}
\def\labelenumi{\arabic{enumi}.}
\tightlist
\item
  Load RR functions
\end{enumerate}

\begin{Shaded}
\begin{Highlighting}[]
\KeywordTok{summary}\NormalTok{(cars)}
\end{Highlighting}
\end{Shaded}

\begin{verbatim}
##      speed           dist       
##  Min.   : 4.0   Min.   :  2.00  
##  1st Qu.:12.0   1st Qu.: 26.00  
##  Median :15.0   Median : 36.00  
##  Mean   :15.4   Mean   : 42.98  
##  3rd Qu.:19.0   3rd Qu.: 56.00  
##  Max.   :25.0   Max.   :120.00
\end{verbatim}

\begin{enumerate}
\def\labelenumi{\arabic{enumi}.}
\setcounter{enumi}{1}
\tightlist
\item
  Make the calculations of \(\Delta\) DALYs
\end{enumerate}

\hypertarget{assessing-age-sex-country-micronutrient-deficiences-at-present-and-in-2030-using-sevs}{%
\subsubsection{Assessing age-sex-country micronutrient deficiences at
present and in 2030 (using
SEVs)}\label{assessing-age-sex-country-micronutrient-deficiences-at-present-and-in-2030-using-sevs}}

\begin{enumerate}
\def\labelenumi{\arabic{enumi}.}
\tightlist
\item
  The burden of health of vitamin A and zinc occurs according to the GBD
  at ages 1-5. However to assess the prevalence of micronutrient
  deficiency we can use the SEV defined above. For the purpose of this
  calculation are in fact the prevalence of micronutrient deficiency
  using the probability method. It however requires to derive continuous
  RR curves, rather than the dichotomous RR for vitamin A and zinc used
  in the GBD methodology.
\end{enumerate}

Therefore the change (in percentage) in micronutrient prevalence for
each sex-age-country group and micronutrient r following a perturbation
(increase/decrease consumption of fish) will be

\[
                                    ∆SEV_{c,a,s,r}=(SEV^{alt}_{c,a,s,r}/SEV^{ref}_{c,a,s,r}-1)*100
                                    \]

\end{document}
